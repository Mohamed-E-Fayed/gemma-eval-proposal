\documentclass[
	%a4paper, % Use A4 paper size
	letterpaper, % Use US letter paper size
]{jdf}

\addbibresource{references.bib}

\author{Mohamed Fayed}
\email{mohamed.fayed.425@gmail.com}
\title{Benchmarking Gemma Models}

\begin{document}
%Benchmark Gemma Models: Develop a comprehensive benchmark suite to test Gemma models on a range of tasks and datasets (academic benchmarks like MMLU, GSM8K, etc., and custom datasets).
%
%Create scripts for automation. Automate the benchmarking process for various tasks.
%Compare performance of various Gemma model families. Run the benchmarks on different Gemma model sizes and variants.
%Compare performance against other open models. Include other popular open models (e.g., Llama 2, Mistral) in the benchmark for comparison.
%Create informative summaries of benchmark results. Generate reports and visualizations summarizing the results. This should include tables, charts, and potentially a leaderboard.
%Regular Updates: Design the benchmark to be easily updated with new Gemma models and datasets.
%Reproducibility: Provide clear instructions and scripts to allow others to reproduce the benchmark results.
%Complexity: Medium to High. Requires understanding of benchmarking methodologies, machine learning evaluation, and scripting.
%
%Expected Size: 175-350 hours.
%
%Skills: Python, machine learning evaluation, scripting, data analysis.

\begin{abstract}
Gemma models has shown impressive capabilities on many benchmarks.
However, the authors did not test other open source models against Gemma because their setting does not guarantee for fairness among all of them. %check report page 5.
    Moreover, there are more specialized benchmarks on which the performance of Gemma is yet to be determined.
    In our proposal for GSoC 2025, we suggest coding a tool for evaluating a set of Text only and Multimodal Large Language Models.
     \end{abstract}

\section{Introduction}
% landscape
% problems
% contributions
During the course of this project, I aim at developping a framework to evaluate Multimodal Large Language Models of the following features:
     \begin{itemize}
              \item Support for diverse set of benchmark datasets,
              \item Build an interface for Huggingface Models to support many models including Gemma, LLaMA, Mistral and other models released in the future.
              \item Easy interface to run on benchmark datasets, and
              \item A leaderboard including all results for models tested with our framework.
                   \end{itemize}

\section{Benchmarks}\label{sect:benchmarks}
There has been a lot of benchmarks introduced to test MLLMs.
So, I select a sample of them to be representative to many categories from development perspective.
For other benchmark testsets, they are expected to be added at a later stage to test the modularity of the framework.
\subsection{Commonly Used Benchmarks}\label{ssect:common}
This is a list of commonly used benchmarks in reporting LLMs/MLLMs performance:
\begin{enumerate}
    \item MMLU-PRO~\cite{wang2024mmlu} and/or MMLU-ProX~\cite{xuan2025mmlu}
    \item TruthfulQA\cite{lin2021truthfulqa}
    \item HellaSwag~\cite{zellers2019hellaswag} and/or HellaSwag-Pro~\cite{li2025hellaswag}
    \item Big-Bench Lite~\cite{srivastava2023beyond}
    \item IFEval~\cite{zhou2023instruction} and/or IFEval-Extended~\cite{kovalevskyi2024ifeval}
              \end{enumerate}
              \subsection{Coding Tasks}
              \begin{itemize}
    \item CodeXGLUE~\cite{lu2021codexglue}
                            \end{itemize}
                            \subsection{Chart Related Tasks}
\subsubsection{Chart-to-Table}
\begin{itemize}
    \item Testsets: ChartQA~\cite{masry2022chartqa}, PlotQA~\cite{methani2020plotqa}, ICPR22
         \item Metrics: Relative Number Set Similarity and Relative Mapping Similarity~\cite{liu2022deplot}
              \end{itemize}
\subsubsection{Chart Question Answering}
\begin{itemize}
    \item Testsets: ChartQA~\cite{masry2022chartqa}, PlotQA~\cite{methani2020plotqa}, ICPR22
         \item Metrics: Accuracy, Precision, Recall and F1
              \end{itemize}
\subsubsection{Chart Summarization}
\begin{itemize}
    \item Testsets: Chart-to-Text~\cite{kantharaj2022chart} and ChartSumm~\cite{rahman2023chartsumm}
    \item Metrics: BLEU~\cite{post2018call}, CIDEr~\cite{vedantam2015cider}, ROUGE~\cite{lin2004rouge} and BLEURT~\cite{sellam2020bleurt}.
              \end{itemize}
\subsection{LLMs as Agents}
\begin{itemize}
    \item Software Engineering Agents: SWE-Bench~\cite{yang2024swe}
    \item Machine Learning Researchers: MLAgentBench~\cite{huang2023mlagentbench}, MLGym~\cite{nathani2025mlgym}
              \end{itemize}
\subsection{Task Selection}
The following is the list of selected benchmark to implement: TBD
\section{Technical Implementation Plan}
\begin{table}
 \begin{tabular}{|c|c|c|}
     Phase & Starts & Ends \\
     MLLM Evaluation Core & 1 & 2 \\
     Add support for other models and datasets (new categories) & 3 & 8 \\
     Implement Leaderboard Interface &  5 & 8 \\
     Add support for other models and datasets (with no new categories) & 9 & 12 \\
 \end{tabular}
 \caption{Main phases Timeline in terms of weeks.}\label{tab:timeline}
  \end{table}

\section{Anticipated Impact}\label{sect:impact}
By creating this MLLMs evaluation framework, we make it easier to evaluate and compare among MLLMs and understand their capabilities in fine-grained details in domain specific tasks.

\section{References}
\printbibliography[heading=none]

\end{document}
